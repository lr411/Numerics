%Chapter 1 FTBS method

%\chapter{FTBS}

The first method to be explored is FTBS. Characteristics of the method can be found in \cite{mpebook}.
FTBS has been chosen as "naive" method to be compared to a more "sophisticathed" method and show the quantitative and qualitative differences.

\section{General characteristics}
We will explore the characteristics of the FTBS method. Unless otherwise stated, the results used in this chapter will be taken without proof from \cite{mpebook} and \cite{nmnotes}.

\subsection{Conservation of mass} \label{conservationftbs}
With reference to equation (\ref{eq:linadvec_initcondition}), the mass of $\phi$ is conserved under the linear advection in the exact solution and also in the FTBS numerical implementation, whereas the variance of $\phi$ is conserved only in the exact solution but not in the FTBS implementation. The variance decreases in FTBS, and this is consistent with what we will find in the subsection \ref{stabilityftbs}, that the method is damping.

\subsection{Stability} \label{stabilityftbs}


\subsection{Accuracy}
The FTBS method is first order accurate

\subsection{Monotonicity}

\subsection{Dispersion errors}

\subsection{Diffusion errors}

\subsection{Computational modes}
What are these??? I read they refer to nothing specifically, but in general to modes that can be either physical or coming from numerical implementations. Not sure I understood how to compute these.

\subsection{Computational Cost}

\subsection{Variable resolution}

