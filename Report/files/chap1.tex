%Chapter  Numerical methods

\section{Numerical Methods}
All the equations of the numerical methods presented in this chapter have been taken from \cite{mpebook} and \cite{nmnotes}. Meaning of symbols, when not explicitly explained, are assumed also as in \cite{mpebook} and \cite{nmnotes}. Additional sources will be acknowledged when used throughout the chapter.
Both time and space have been assumed to vary in [0,1], and the number of space points has been kept equal to the number of time points in the simulation, for simplicity and to allow comparisons among multiple runs, avoiding rounding errors.

\subsection{FTBS}

\begin{equation}
\phi_j^{n+1}=\phi_j^{n}-c(\phi_j^{n}-\phi_{j-1}^{n})
\label{eq:ftbs}
\end{equation}
Equation (\ref{eq:ftbs}) describes the time evolution of FTBS. As described in \cite{mpebook}, the scheme is first order accurate in time and space, and stable for $c\in[0,1]$. It conserves mass but not variance. it has been included in this report for it's "naivety", to be used as term of comparison with other, better performing schemes.


\subsection{CTCS}
\begin{equation}
\phi_j^{n+1}=\phi_j^{n}-c(\phi_{j+1}^{n}-\phi_{j-1}^{n})
\label{eq:ctcs}
\end{equation}
Equation (\ref{eq:ctcs}) describes the time evolution of CTCS. As described in \cite{mpebook}, the scheme is second order accurate in time and space, and stable for $c\in[-1,1]$. The method introduces some computational modes, and conserves mass and variance.


\subsection{LaxWendroff}
\begin{equation}
\phi_j^{n+1}=\frac{c}{2}(c+1)\phi_{j-1}^{n}+(1-c^2)\phi_{j}^{n}+\frac{c}{2}(c+1)\phi_{j+1}^{n}
\label{eq:laxwendroff}
\end{equation}
Equation (\ref{eq:laxwendroff}) describes the time evolution of LaxWendroff method. The method is second order accurate in space and time and, does not add spurious computational modes: in the LaxWendroff case (unline CTCS), in the CFL criterion analysis, the domain of convergence study of the numerical scheme leaves no "open" points, there is only one physical solution. It can be shown, that the method is stable for $c\in[-1,1]$, in fact it can be shown, with some calculations, that the amplification factor in the Von Neumann analysis is:
\begin{equation}
\label{eq:laxwen_vonneumann}
A=(1+c^2(cos(k\Delta x)-1))-icsin(k\Delta x)
\end{equation}
and the condition on $|A|\leq1$ brings the constraint $c\in[-1,1]$.

\subsection{CNCS}
\begin{equation}
\phi_j^{n+1}=\phi_j^{n}-\frac{c}{4}(\phi_{j+1}^{n+1}-\phi_{j-1}^{n+1}+\phi_{j+1}^{n}-\phi_{j-1}^{n})
\label{eq:cncs}
\end{equation}
Equation (\ref{eq:cncs}) describes the time evolution of CTCS. As described in \cite{mpebook}, the scheme is second order accurate in time and space, and stable for $c\in\mathbb{R}$.

