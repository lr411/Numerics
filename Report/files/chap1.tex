%Chapter 1 FTBS method

%\chapter{FTBS}

The first method to be explored is FTBS. Characteristics of the method can be found in \cite{mpebook}.
FTBS has been chosen as "naive" method to be compared to a more "sophisticathed" method and show the quantitative and qualitative differences.

\section{General characteristics}
We will explore the characteristics of the FTBS method.

\subsection{Conservation of mass}
FTBS conservation properties are shown in paragraph 5.2.9 \cite{mpebook} and 4.7, 4.7.1 of \cite{nmnotes}. With reference to equation (\ref{eq:linadvec_initcondition}), the mass of $\phi$ is conserved under the linear advection in the exact solution and also in the FTBS numerical implementation, whereas the variance of $\phi$ is conserved in the exact solution but not in the FTBS scheme

\subsection{Stability}

\subsection{Accuracy}

\subsection{Monotonicity}

\subsection{Dispersion errors}

\subsection{Diffusion errors}

\subsection{Computational modes}
What are these??? I read they refer to nothing specifically, but in general to modes that can be either physical or coming from numerical implementations. Not sure I understood how to compute these.

\subsection{Computational Cost}

\subsection{Variable resolution}


\section{Phyton code}
The function used for the method is in the file advectionSchemes.py, and the function in the code is the following, with comments included on the meaning of inputs:
\begin{lstlisting}
def FTCS(phiOld, c, nt):    
"Linear advection scheme using FTCS, with Courant number c and"
"                       nt time-steps"
"inputs are:"
"phiOld: initial condition on phi (to save space the array will then"\
"               be used to store values from the previous time step)"
"c: Courant number"
"nt: nr of time steps"
\end{lstlisting}



