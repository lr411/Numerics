%Chapter 1 FTBS method

%\chapter{FTBS}



The first method to be explored is FTBS. Characteristics of the method can be found in \cite{mpebook}.
FTBS has been chosen as "naive" method to be compared to a more "sophisticathed" method and show the quantitative and qualitative differences.

\section{General characteristics}
We will explore the characteristics of the FTBS method, in terms of 


\section{Phyton code}
The function used for the method is in the file advectionSchemes.py, and the function in the code is the following, with comments included on the meaning of inputs:
\begin{lstlisting}
def FTCS(phiOld, c, nt):    
"Linear advection scheme using FTCS, with Courant number c and"
"                       nt time-steps"
"inputs are:"
"phiOld: initial condition on phi (to save space the array will then"\
"               be used to store values from the previous time step)"
"c: Courant number"
"nt: nr of time steps"
\end{lstlisting}



