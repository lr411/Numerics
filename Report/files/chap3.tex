%Chapter Python


\section{General characteristics}
In this section we will explain the python implementation of the numerical schemes. 
Currently the implementation consists of 4 files, each implementing one of the following:
\begin{itemize}
	\item Main file used to run the project
	\item Initial condition file, stores different possible choices of initial conditions that can be used to test different properties
	\item Advection schemes file
	\item Diagnostics functions file
\end{itemize}

\section{Code Repository}
The code repository is:\\
https://github.com/lr411/Numerics.git\\
The hash of the commit the report refers to for the code part, is:\\
43f4315e753289bfaab8d4c84fa88d7124c732dc


\section{Advection Schemes Implementation}
The function used for the method is in the file advectionSchemes.py, and the function in the code is the following, with comments included on the meaning of inputs:
\begin{lstlisting}
def FTCS(phiOld, c, nt):    
"Linear advection scheme using FTCS, with Courant number c and"
"                       nt time-steps"
"inputs are:"
"phiOld: initial condition on phi (to save space the array will then"\
"               be used to store values from the previous time step)"
"c: Courant number"
"nt: nr of time steps"
\end{lstlisting}
