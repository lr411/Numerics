%:
%\documentclass[12pt,mythesisstyle,twoside]{report}%% for two-sided printing
\documentclass[12pt,mythesisstyle]{report}
\usepackage{files/mythesisstyle}
%\usepackage{bbm}

\usepackage{amssymb,amsmath}
\usepackage{calc}
\usepackage{verbatim}
%\usepackage[mathscr]{eucal}
%\usepackage{pstricks}
\usepackage{latexsym,amsfonts}
  \usepackage{graphicx,graphics}% allows for graphics
\usepackage{float} % floats the figures
%\floatstyle{boxed} % nicely boxes the figures
\restylefloat{figure}
%% use this package to add color to LaTeX
%\usepackage[dvips]{color}
\usepackage{files/setspace}
\usepackage{listings}

%\doublespacing
\onehalfspacing

% my_lib.tex   My Library for LaTeX
% 

\newcommand{\sfrac}[2]{\genfrac{}{}{0.5pt}{1}{#1}{#2}}


% change this if we want italic d's in dy/dx (see diff below)
\newcommand{\dif}{\mathrm{d}}
\newcommand{\di}{\partial}
\newcommand{\bigoh}{\mathcal{O}}
\newcommand{\diff}[2]{\frac{\dif #1}{\dif #2}}
\newcommand{\pdiff}[2]{\frac{\di #1}{\di #2}}
\newcommand{\n}{\mathbf{n}}

% allows me to change between blackboard and funky Q,C,R,N,Z
\newcommand{\Complex}{\mathbb{C}}
\newcommand{\Reals}{\mathbb{R}}
\newcommand{\Naturals}{\mathbb{N}}
\newcommand{\Rationals}{\mathbb{Q}}
\newcommand{\Integers}{\mathbb{Z}}

\newcommand{\program}[1]{\textbf{#1}}
\newcommand{\function}[1]{\texttt{#1}}
\newcommand{\file}[1]{\texttt{#1}}
\newcommand{\para}[1]{\texttt{#1}}
% use these for the names of languages
\newcommand{\fortran}{\texttt{FORTRAN}}
\newcommand{\matlab}{\texttt{MATLAB}}
\newcommand{\cpp}{\texttt{C++}}
\newcommand{\perl}{\texttt{PERL}}

\newcommand{\matlabtm}{\textsc{Matlab}}
\newcommand{\octavetm}{\textsc{Octave}}
\newcommand{\scilabtm}{\textsc{SciLab}}

% scientific notation:
\newcommand{\scin}[2]{$#1 \times 10^{#2}$}
\newcommand{\scinm}[2]{#1 \times 10^{#2}}

% this makes eqnarray not have funny spaces around the =
%\setlength\arraycolsep{2pt}

% using report without chapters...
%\renewcommand{\thesection}{\arabic{section}}








  %%  custom commands
\renewcommand{\Re}{\text{Re}}
\renewcommand{\Im}{\text{Im}}

%%% Depth of table of contents 3- for subsubsections,
\setcounter{tocdepth}{2}
%\setcounter{secnumdepth}{3}


%% sets things up for running PDFLaTeX / LaTeX
%%\newif\ifpdf
  %%\ifx\pdfoutput\undefined
  %%\pdffalse    % we are not running PDFLaTeX
%%\else
  %%\pdfoutput=1 % we are running PDFLaTeX
  %%\pdftrue
%%\fi
%%\ifpdf
  % if we're compiling with PDFLaTeX then use .pdf, .jpg or .tif for figures
%  \DeclareGraphicsExtensions{.pdf, .jpg, .tif}
%%\else
  % if we're not using PDFLaTeX, then use the .eps or .ps versions of figures
%%  \usepackage{graphicx}
 %  \DeclareGraphicsExtensions{.eps, .ps}
%%\fi

%% Use this for draft printouts (ex: to submit to supervisor)
\newcommand{\draftmarker}{\textsc{draft version: \ifnum \day<10 0\fi \number\day/\ifnum \month<10 0\fi \number\month/\number\year}}

%% Uncomment this block of code to get a draft marker on each page
%% this has the unfortunate side effect that you get a overful vbox
%% warning of 2.4pts for every page in your thesis.  My policy was turn
%% it on before making a drift printout (that way every page will have
%% the date and say draft on it).
%% place a ``%'' in front of the next line to enable draft marking:
\begin{comment}
    \usepackage{fancyhdr}
    \fancypagestyle{plain}{
       \fancyhf{}
       \fancyfoot[C]{\thepage}
       \fancyfoot[R]{\draftmarker}
    }
    \pagestyle{fancy}
    \lhead{}  \chead{}   \rhead{}
    \lfoot{}  \cfoot{\thepage}    \rfoot{\draftmarker}
    \renewcommand{\headrulewidth}{0pt}
    \renewcommand{\footrulewidth}{0pt}
%% place a ``%'' in front of the next line to enable draft marking:
\end{comment}
%% **** end draft stuff ****

\begin{document}
%%% set switches
%\contentspagefalse
%\figurespagetrue
\tablespagetrue
%\dedicationpagetrue
%\quotationpagetrue
%\otherlistpagetrue

%%% front matter


% *********************************************************************
            % TITLE
% *********************************************************************
\title{Numerical Methods \\
        Course Assignment Report \\  % or change to something similar
        }

\author{Leonardo Ripoli}

%\docrevision{1 0}

\qualification{}

%\degree{Master of Science}
%\dept{Mathematics}
    % Semester & year you submit dissertation to the GSAS.
\submitdate{15 November 2017}

\advisor{Hilary Weller}

    %Optional title page items: remove percent if you need them.
%\entity{School}    % Department (default), School, etc.
%\endeavour{thesis}  % thesis (default),extended essay, etc.
\copyrightyear{2017}   % year LaTeX'ed (default).




\beforepreface



% *********************************************************************
            % ABSTRACT, ACKNOWLEDGMENTS, ETC.

% *********************************************************************

    % Use \prefacesection for Abstract, Acknowledgments,
    % Dedication, Preface, etc. sections

            % A NOTE ABOUT LINE SPACING

% This style uses linespacing that is 1.3 times normal ("singlespace"),
% except in footnote, \begin{table} and \begin{figure} which
% use normal linespacing. To modify linespacing (eg, to 1.6 x normal),
 %      \renewcommand{\baselinestretch}{2} \tiny\normalsize
% These commands have to be used here, not in the preamble.

%\prefacesection{Dedication}
%\input{files/dedication}

%\prefacesection{Acknowledgments}
%\input{files/ackno}

\prefacesection{Abstract}
% Abstract goes here

In this work we present the analysis of the linear advection equation modelled in one dimension, x, without sources or sinks of the advected variable $\phi$. The exact expression of the equation is:
\begin{equation} \label{eq:linadvec}
\phi_{t}+u\phi_{x}=0
\end{equation}
We consider the case of constant and uniform wind, u, and with given initial condition $\phi(x,0)=\phi_{0}$.
It can be shown that the analytic solution of \ref{eq:linadvecq} is:
\begin{equation} \label{eq:linadvec_initcondition}
\phi(x,t)=\phi_{0}(x-ut)
\end{equation}
We have modelled equation \ref{eq:linadvecq} using several numerical schemes:

used for its characteristic of higher order of convergence (2) compared\lessapprox


\afterpreface

\onehalfspacing


% *********************************************************************
% THE CHAPTERS
% *********************************************************************

\chapter{FTBS}\label{chap:chap1}
%Chapter 1 FTBS method

%\chapter{FTBS}

The first method to be explored is FTBS. Characteristics of the method can be found in \cite{mpebook}.
FTBS has been chosen as "naive" method to be compared to a more "sophisticathed" method and show the quantitative and qualitative differences.

\section{General characteristics}
We will explore the characteristics of the FTBS method. Unless otherwise stated, the results used in this chapter will be taken without proof from \cite{mpebook} and \cite{nmnotes}.

\subsection{Conservation of mass} \label{conservationftbs}
With reference to equation (\ref{eq:linadvec_initcondition}), the mass of $\phi$ is conserved under the linear advection in the exact solution and also in the FTBS numerical implementation, whereas the variance of $\phi$ is conserved only in the exact solution but not in the FTBS implementation. The variance decreases in FTBS, and this is consistent with what we will find in the subsection \ref{stabilityftbs}, that the method is damping.

\subsection{Stability} \label{stabilityftbs}


\subsection{Accuracy}
The FTBS method is first order accurate

\subsection{Monotonicity}

\subsection{Dispersion errors}

\subsection{Diffusion errors}

\subsection{Computational modes}
What are these??? I read they refer to nothing specifically, but in general to modes that can be either physical or coming from numerical implementations. Not sure I understood how to compute these.

\subsection{Computational Cost}

\subsection{Variable resolution}



%% Bibliography. I didn't use BibTeX, so you're on your own if you'd like to do so.
%
\begin{thebibliography}{99}\addcontentsline{toc}{chapter}{Bibliography}

\bibitem{mpebook}
J. Br\"{o}cker, B. Calderhead, D. Cheraghi, C. Cotter, D. Holm, T. Kuna, B. Pelloni, T. Shepherd, H. Weller.
\newblock M$\alpha$thematics of Planet Earth
\newblock {\em World Scientific}, 2017



\end{thebibliography}


\end{document}




